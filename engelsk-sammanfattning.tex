In this Bachelor’s thesis the following question is answered: Does the
inequality posed in the article Klyachko et al [2008] cover the real
part of the Bloch surface of a 3D quantum system when used as in Kochen
and Specker [1967]? The Klyachko inequality relies on using five
measurements to show contextuality of a subset of states on the real
part of the Bloch surface. These can now be used in several
configurations as present in the Kochen-Specker contextuality proof, by
simply rotating the measurements. We show here that these new
inequalities will have subsets of violation that eventually cover the
entire real part of the Bloch surface. This can be extended to show that
all states of a spin 1 system are non-contextual, so that we have
recovered a state-independent contextuality proof by using the Klyachko
inequality several times. In the final part, an interpretation of this is
given and also some recommendations for further research that should be
done in the field.