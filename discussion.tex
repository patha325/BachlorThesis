\chapter{Discussion}\label{cha:discussion}
\newpage
\section{Was the sphere covered by inequalities?}
As seen in figure~\ref{fig:coveredsphere} and discussed in~\subsectionref{Is the sphere covered?} it seems that the whole sphere is covered by these inequalities. Through the whole thesis there have been signs and proof that this may be the case, however it was only fully seen at the end. What does this imply? 
\subsection{What does this imply?}
If the whole Bloch sphere is either inside or on the border of the inequalities as seen in the thesis, this means that all possible states of a spin 1 system can be explained by the details given in the article~\cite{PhysRevLett.101.020403}. This means that the inequality posed is always violated  and that quantum mechanics can be used to explain the spin 1 system.
Two questions arise from this.\\
It says all states above, but is this really true?\\
Has it then been shown that all possible states of a spin 1 system can be explained by the details given in the article~\cite{PhysRevLett.101.020403}?\\
The answer to both of these questions is no.\\
There were a few restrictions given in the start of this thesis. Remember that to be able to visualize the states simply the Bloch sphere was used. However it can not be used for a spin 1 system, unless only real coefficients for the vectors that represent the states are used. This was a restriction that has been used throughout the thesis. It is quite undeniable that a proof that the inequality covers all of the real states has been found in this thesis, however nothing has been said about those which require a complex representation. Therefore, it can be said that the non-contextual inequality given in the article~\cite{PhysRevLett.101.020403} is violated for all of the real states that exist for a spin 1 system. To say anything else would require more research.

